\section{Einleitung}

Als unerfahrener Gärtner oder Gärtnerin ist man gezwungenermaßen auf das Wissen und die Erfahrungen anderer angewiesen, wenn es um Fragen wie das Anlegen von Beeten, Pflanzpläne, Pflege von Beerenobst oder Obstbäumen, Ernten oder Lagern geht.
Dieses Wissen ist in der Regel in Büchern zusammengefasst, kann seit ein paar Jahren aber auch in Podcasts, Videos, Blogs oder auf Websites gefunden werden.
Seit jeher aber natürlich auch in Gesprächen mit erfahrenen Gärtnern.

Das führt dazu, dass spezifisches Wissen in der Regel auf verschiedene Quellen verstreut ist und dass man daher oft nach kurzer Zeit nicht mehr weiß, was man wo gelesen oder gehört hat.
Die Motivation dieses Buches ist es daher, Wissen zum Gärtnern ``zielgerichtet" zusammenzufassen und die entsprechenden Quellen zu erfassen.
Zielgerichtet heißt in diesem Fall, dass vorrangig die Pflanzen und Methoden berücksichtigt werden, die für unseren Garten im Gewann Hohlweg der Gartenfreunde Heslach relevant sind.
Das Ziel ist, ein Nachschlagewerk zu schaffen, das die wichtigsten Aspekte zusammenfasst und auch spontan, d.h. vor Ort, in seiner Onlineversion verfügbar ist.

Das Buch ist in drei Teile gegliedert, wobei der erste Teil ``Jahreszeiten" monatsweise eine Übersicht geben soll, was wann wo im Garten zu tun ist.
In diesem Teil finden sich Verweise auf die konkreten Beschreibungen in den Teilen II und III zum ``Nutzgarten" und respektive ``Ziergarten".

Die Liste der untersuchten Quellen ist noch relativ kurz, genauso wie die Zeit seit 2018, in der wir bisher als Gärtner eigene Erfahrungen sammeln konnten.
Die Hoffnung ist, dass dieses Buch zusammen mit der Anzahl seiner Quellen und unseren eigenen Erfahrungen, gesund wachsen wird.

