\part{Jahreszeiten}

\pagebreak

\section{Frühjahr}

\subsection{Januar}

Im Januar können bei frostfreiem Wetter Schnittarbeiten an \textrightarrow Bäumen (siehe \ref{Bäume}) und \textrightarrow Sträuchern (\ref{Sträucher}) durchgeführt werden.
Stangl \cite[S.~256]{Stangl1995} spricht vom ``Auslichten" der Ziersträucher und älterer Bäume.

Weiterhin ist der Januar ein guter Monat, um einen Anbauplan für die Gemüsebeete anzufertigen.
Dabei sollten die Erfahrungen des vorhergehenden Jahres berücksichtigt werden, d.h. von Gemüsearten mehr oder weniger einplanen, je nachdem wie ertragreich das Jahr war \cite[S.~256]{Stangl1995}.
Ebenfalls empfiehlt es sich in diesem Monat die Keimfähigkeit des vorhandenen Saatguts zu prüfen und ggf. neu zu kaufen \cite[S.~216]{Heberer2018}.
Zur richtigen Lagerung von Saatgut siehe Heberer \cite[S.~179]{Heberer2018}.

\subsection{Februar}

...

\subsection{März}
Heberer \cite[S.~11]{Heberer2018} empfiehlt zur Bodenvorbereitung und -verbesserung eine sogenannte Gründüngung für solche Beete, in die erst nach den Eisheiligen (Mitte Mai) die forstempfindlichen Blumen oder Gemüsepflanzen einziehen.
Zum prinzipiellen Vorgehen und zu möglichen Pflanzenarten, die für die Gründüngung geeignet sind, siehe Heberer \cite[S.~11]{Heberer2018} und \cite[S.~114f]{Heberer2018}.

\subsection{April}

...

\subsection{Mai}

Gegen Mitte bzw. Ende des Monats wird es gewöhnlich nochmal kalt (Eisheiligen).
Das heißt, dass forstempfindliche Blumen oder Gemüsepflanzen erst danach in die Beete einziehen sollten \cite[S.~11]{Heberer2018}.



\pagebreak
