\part{Nutzgarten}

\pagebreak

\section{Gemüse}

\subsection{Anbauplan}
...

\section{Strauchbeerenobst}

\subsection{Himbeere}
\label{Himbeere}

Nach Don \cite[S.~420]{Don2021} unterscheidet man zwischen Sommer- und Herbsthimbeeren.
Erstere tragen von Juni bis August.
Ihre Tragzeit überschneidet sich um rund eine Woche mit den Herbsthimbeeren, die je nach Wetter ab August bis in den Oktober hinein geerntet werden können.

Herbsthimbeeren werden in größeren Abständen als Sommerhimbeeren gepflanzt, da sie keine Ruten, sondern Büsche bilden und die Früchte am diesjährigen Wuchs tragen.
Daher müssen sie auch anders als Sommerhimbeeren geschnitten werden.
Sobald sie ihr Laub abgeworfen haben, was um Weihnachten der Fall ist, schneidet man ihren gesamten oberirdischen Wuchs bis auf den Boden zurück, damit die Pflanze im Frühjahr komplett neu austreibt \cite[S.~421]{Don2021}.

Himbeeren bevorzugen zwar hohe Luftfeuchtigkeit und viel Regen, verabscheuen aber kalte, nasse Böden.
Insbesondere sollte verhindert werden, dass sie im Winter im Wasser stehen.
Ebenfalls vertragen Sommer- wie Herbsthimbeeren keine Trockenheit.
Eine dicke Mulchschicht im Frühjahr hält die Erde des Flachwurzlers feucht und kühl.
Wir haben die Erfahrung gemacht, dass eine dicke Schicht aus gehäckseltem Schnittabfall ideal ist \cite[S.~421]{Don2021}.


\subsection{Johannisbeere}
...

\subsection{Brombeere}
...

\section{Obstbäume}
\label{Bäume}

Nach Stangl \cite[S.~256]{Stangl1995} gilt es beim Auslichten älterer Bäume vorrangig kranke, dürre oder zu dicht stehende Äste zu entfernen.

\subsection{Kirschbaum}
...



\pagebreak