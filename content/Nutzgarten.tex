\part{Nutzgarten}

\pagebreak

\section{Gemüse}

\subsection{Anbauplan}
...

\pagebreak

\section{Strauchbeerenobst}
\label{Strauchbeerenobst}

\subsection{Allgemeines}

\paragraph{Mulchen}

Heberer \cite[S.~44]{Heberer2018} rät nach den Eisheiligen im \textrightarrow Mai (\ref{Mai}) unter allen Beerensträuchern und -hecken eine Mulchdecke auszubringen.
Die Erde ist nun gut erwärmt und ab jetzt ist es wichtig, den Boden feucht zu halten und Unkrautbewuchs zu unterdrücken.

Zum Mulchen von Beerensträuchern sollte im Mai zunächst eine Schicht Kompost unter den Pflanzen ausgebracht werden.
Im weiteren Verlauf des Sommers dann Rasenschnitt, der ohnehin bei jedem Mähen anfällt.
Die Mulchschicht sollte gleichmäßig dick sein und mehrmals erneuert werden, so dass der Boden ständig bedeckt, feucht und humusreich ist.
Zudem bleibt der Boden locker ohne dass gehackt werden muss.
Dieser Aspekt ist gerade bei Beerensträuchern wichtig, da ihre Wurzeln hauptsächlich flach verlaufen und bei jeder Bodenbearbeitung beschädigt werden könnten.
Rasenschnitt, sowie auch Stroh und Laub, lässt sich meist am einfachsten mit den Händen verteilen \cite[S.~45]{Heberer2018}.

\subsection{Himbeere}
\label{Himbeere}

Nach Don \cite[S.~420]{Don2021} unterscheidet man zwischen Sommer- und Herbsthimbeeren.
Erstere tragen von Juni bis August.
Ihre Tragzeit überschneidet sich um rund eine Woche mit den Herbsthimbeeren, die je nach Wetter ab August bis in den Oktober hinein geerntet werden können.

\paragraph{Pflanzen}

Sommerhimbeeren werden nach Stangl \cite{Stangl1995} im Abstand von 40~cm, nach Don \cite[S.~420]{Don2021}) im Abstand von 60~cm gepflanzt.
Herbsthimbeeren sollten jedoch in größeren Abständen als Sommerhimbeeren gepflanzt werden, da sie keine Ruten, sondern Büsche bilden \cite[S.~420]{Don2021}.
Welcher Abstand konkret gewählt werden sollte, ist aus den o.g. Quellen nicht ersichtlich.

\paragraph{Schneiden}

Da Herbsthimbeeren die Früchte am diesjährigen Wuchs tragen, müssen sie auch anders als Sommerhimbeeren geschnitten werden.
Sobald sie ihr Laub abgeworfen haben, was um Weihnachten der Fall ist, schneidet man ihren gesamten oberirdischen Wuchs bis auf den Boden zurück, damit die Pflanze im Frühjahr komplett neu austreibt \cite[S.~421]{Don2021}.

\paragraph{Gießen}

Himbeeren bevorzugen zwar hohe Luftfeuchtigkeit und viel Regen, ``verabscheuen" aber kalte, nasse Böden.
Insbesondere sollte verhindert werden, dass sie im Winter im Wasser stehen.
Ebenfalls vertragen Sommer- wie Herbsthimbeeren keine Trockenheit.
Eine dicke Mulchschicht im Frühjahr hält die Erde des Flachwurzlers feucht und kühl.
Nach Don \cite[S.~421]{Don2021} ist eine dicke Schicht aus gehäckseltem Schnittabfall ideal.

\paragraph{Mulchen}

Siehe Allgemeines zum Strauchbeerenobst.

\subsection{Johannisbeere}
...

\subsection{Brombeere}
\label{Brombeere}

Brombeeren fruchten am zweijährigen Holz.
D.h. im Winter werden alle diejenigen Ruten / Ranken herausgeschnitten, die im vorherigen Jahr Früchte getragen haben \cite[S.~176]{Seymour1978}.
Konkret wird dieser ``Winterschnitt" allerdings erst im Frühjahr nach der Frostperiode durchgeführt \cite[S.~196]{Stangl1995}.

\paragraph{Schneiden}

Nach Stangl \cite[S.~196]{Stangl1995} ist die wichtigste Arbeit bei der Pflege von Brombeeren der Sommerschnitt:
``Versäumen wir ihn, so bildet sich in kurzer Zeit ein Triebgewirr, in dem wir uns kaum mehr zurecht finden."

Beim \textit{Sommerschnitt} werden die sogenannten vorzeitigen Triebe oder Geiztriebe, die während des Sommers aus den Blattachseln entstanden sind, auf ein Blatt eingekürzt sobald sie eine Länge von ca. 50 cm erreicht haben.

Beim \textit{Winterschnitt} werden alle im letzten Jahr mit Beeren behangenen Triebe entfernt.
Von den entstandenen Jungtrieben werden nur die 6 kräftigsten belassen und an den Drähten zu beiden Seiten der Pflanze angebunden.
Die gleichmäßig an den Drähten befestigten Jungtriebe bringen Ertrag, während aus dem Wurzelstock neue Triebe herauswachsen.
Von diesen werden wiederum nur die 6 kräftigsten ausgewählt und an den noch freien Drähten festgebunden \cite[S.~197]{Stangl1995}.

Im Ergebnis hat man dann immer 6 Ranken aus dem Vorjahr, die tragen, und 6 junge Triebe, die während des Sommers aus dem Wurzelstock nachwachsen und auch am Spalier angebunden werden sollen.

\paragraph{Ernten}

Die im Vorjahr entstandenen Ranken werden abgeerntet, sobald die Früchte richtig schwarz geworden sind bzw. wenn sie so reif sind, dass sie beim Pflücken fast von selbst abfallen \cite[S.~176]{Seymour1978}.
Die abgetragenen Ranken bleiben den Winter über am Spaliergerüst, da sie den jungen Trieben etwas Schutz geben, und werden erst im Frühjahr entfernt (\textrightarrow Winterschnitt) \cite[S.~197]{Stangl1995}.

\pagebreak

\section{Obstbäume}
\label{Bäume}

Nach Stangl \cite[S.~256]{Stangl1995} gilt es beim Auslichten älterer Bäume vorrangig kranke, dürre oder zu dicht stehende Äste zu entfernen.

\subsection{Kirschbaum}
...



\pagebreak