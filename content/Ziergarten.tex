\part{Ziergarten}

\pagebreak

\section{Sträucher}
\label{Sträucher}

\subsection{Allgemeines}

\subsubsection{Pflanzen}

...

\subsubsection{Schneiden}

Laut Don \cite[S.~257]{Don2021} gilt beim Schneiden von Sträuchern generell, dass ``immer bis auf ein Blatt, eine Knospe oder einen Ansatz" zurück gekürzt werden sollte.
Ebenso rät Stangl \cite[S.~256]{Stangl1995} beim ``Auslichten" von Zierstäuchern, ältere, zu dicht stehende Triebe über dem Boden abzuschneiden oder aber auf Jungtriebe zurückzusetzen.

Generell sollten Gehölze stets mit einer gut geschliffenen Gartenschere geschnitten werden \cite[S.~257]{Don2021}.

\subsection{Flieder}

Je nach Alter des Fliederstrauchs und Jahreszeit können bzw. sollten unterschiedliche Schnitte durchgeführt werden.
Nach Siemens \cite{Siemens2021} wird z.B. bei jungen Fliedern im Frühjahr oder Herbst ein Erziehungsschnitt durchgeführt bzw. bei alten Sträuchern ein Verjüngungsschnitt.
Desweiteren sollte nach der Blütezeit, d.h. frühesten Ende Mai, ein Erhaltungsschnitt durchgeführt werden.

Nach Monty Don \cite[S.~258]{Don2021} liegt der Reiz von Fliederbüschen in erster Linie in ihren Blüten, die sowohl einzeln wie auch in Sträußen ``einfach grandios" aussehen.
Damit sie möglichst schöne und große Blüten bilden, sollten alte Triebe gleich nach dem Verblühen bis zum Boden zurückgeschnitten werden, womit der Verjüngungsschnitt gemeint sein dürfte, und der Strauch großzügig gemulcht und gewässert werden.
Im Folgenden wird daher das Vorgehen bei einem Verjüngungsschnitt beschrieben:

Das Ziel des \textbf{Verjüngungsschnitts} ist es, die Vitalität ``vergreister" Sträucher zu erhöhen und sie zum Blühen anzuregen. 
Dabei wird ein Teil der Hauptäste oder -triebe stark zurückgeschnitten.
Prinzipiell sollte dieser Schnitt auf einen Zeitraum von 2-3 Jahren verteilt werden, damit die Blüte nicht für ein Jahr ausfällt.
Dementsprechend sollte pro Jahr ca. ein Drittel, maximal jedoch die Hälfte der Hauptäste beschnitten werden.
Dabei werden die Hauptäste auf unterschiedlichen Höhen, etwa von Kniehöhe bis dicht über dem Boden abgeschnitten.
Die beschnittenen Äste treiben dann im Laufe der Saison mit zahlreichen neuen Trieben wieder aus, von denen im nächsten Frühjahr jeweils nur zwei bis drei kräftige, gut verteilte Exemplare stehen gelassen werden.
Diese werden wiederum eingekürzt (siehe Erziehungsschnitt), damit sie kräftiger werden und sich gut verzweigen \cite{Siemens2021}. 

Nach Siemens \cite{Siemens2021} werden beim \textbf{Erziehungsschnitt} im Frühjahr oder Herbst alle abgeknickten und schwachen Triebe entfernt und die jungen Triebe um jeweils etwa ein Drittel bis die Hälfte eingekürzt.
Diese Triebe blühen dann zwar nicht, aber dafür bauen sich die jungen Triebe ``von unten schön buschig auf und werden im Alter dann umso prächtiger".

\subsection{Forsythie}

...

\subsection{Rosen}

...


\pagebreak