\documentclass[]{article}

%opening
\title{}
\author{}

\begin{document}

\maketitle

\begin{abstract}

\end{abstract}

\part{Nutzgarten}

\pagebreak

\section{Obst}

\subsection{Kirschbaum}

\section{Gemüse}

\pagebreak

\part{Ziergarten}

\pagebreak

\section{Sträucher}

\subsection{Flieder}

Je nach Alter des Fliederstrauchs und Jahreszeit können bzw. sollten unterschiedliche Schnitte durchgeführt werden.
Nach Siemens \cite{Siemens2021} wird z.B. bei jungen Fliedern im Frühjahr oder Herbst ein Erziehungsschnitt durchgeführt bzw. bei alten Sträuchern ein Verjüngungsschnitt.
Desweiteren sollte nach der Blütezeit, d.h. frühesten Ende Mai, ein Erhaltungsschnitt durchgeführt werden.
Im Folgenden wird das Vorgehen bei einem Verjüngungsschnitt beschrieben:

Das Ziel des Verjüngungsschnitts ist es, die Vitalität ``vergreister" Sträucher zu erhöhen und sie zum Blühen anzuregen. 
Dabei wird ein Teil der Hauptäste oder -triebe stark zurückgeschnitten.
Prinzipiell sollte dieser Schnitt auf einen Zeitraum von 2-3 Jahren verteilt werden, damit die Blüte nicht für ein Jahr ausfällt.
Dementsprechend sollte pro Jahr ca. ein Drittel, maximal jedoch die Hälfte der Hauptäste beschnitten werden.
Dabei werden die Hauptäste auf unterschiedlichen Höhen, etwa von Kniehöhe bis dicht über dem Boden abgeschnitten.
Die beschnittenen Äste treiben dann im Laufe der Saison mit zahlreichen neuen Trieben wieder aus, von denen im nächsten Frühjahr jeweils nur zwei bis drei kräftige, gut verteilte Exemplare stehen gelassen werden.
Diese werden wiederum eingekürzt (siehe Erziehungsschnitt), damit sie kräftiger werden und sich gut verzweigen. 

\subsection{Forsythie}


\pagebreak

\bibliographystyle{acm}
\bibliography{manuscript}





\end{document}
